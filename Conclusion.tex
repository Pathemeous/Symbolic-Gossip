
\section{Conclusion}\label{sec:Conclusion}

This project looked into how we can use the SMCDEL library to better understand and model Gossip. To the first point, we wrote $\texttt{gsi}$, our \textit{Gossip Scene Investigation} function to better understand SMCDEL's Knowledge Scenes as they pertain to The Gossip Problem. To the latter point, we used the existing notion of a Knowledge Transformer in SMCDEL to write a Knowledge Transformer for the Transparent Gossip Problem. Both of these aforementioned processes helped us build a strong understanding of how SMCDEL approaches Gossip, and specifically what makes it so computationally intensive. With this in mind, we tried using a pre-existing optimize function within SMCDEL's library to reduce complexity, as well as writing our own Simple Transformer, to target the blow up in vocabulary that the Classic Transformer implemented in SMCDEL encrues. 

There are two ways we can analyze our work; by considering how fast it is and how correct it is. In Section X (add Benchmarks), we explore the first point. As we saw, the Transparent Transformer and the Simple Transformer both had large improvements on computation time, specifically as the number of calls increased. 

On the other hand, the correctness of our code still has room for improvement. We believe knowing the true differences between what information these transformers codify requires mathematics outside the scope of a programming project. However, it is our belief that the Simple Transformer makes \textit{less} propositions true than the Classic Transformer, and therefore does not tell false truths. 

In terms of further work, readability and correctness is a big focus. The Gossip Problem is a specific example within the area of Dynamic Epistemic Logic (DEL), and is therefore pretty hard to work with from afar. This is part of the reason we wrote the \texttt{gsi} function, and we advise the reader to test its usability by running \texttt{main}. However, although we decode the propositional variables and observations, the state law is still a large BDD, and uninterpretable by the user. Future work could be done to make this more user friendly, perhaps by way of \texttt{graphviz} (Cite graphviz?).

