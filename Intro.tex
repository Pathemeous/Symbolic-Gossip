\section{Introduction}

The \textit{Gossip Problem} is the problem of sharing information in a network.
Many variants of the gossip problem exist, each with their own computational challenges. 
Most notably, a distinction is made between the situation where agents know which agents share 
information and when, known as the \textit{transparent} Gossip Problem and when agents only know that 
information is shared, known as the \textit{synchronous} Gossip Problem.

For simulating the Gossip Problem, an explicit model checker for Gossip called GoMoChe exists \cite{gattinger2023gomoche}.
Explicit model checkers are generally less efficient than symbolic ones, which aim to cut down on computation time.
GoMoChe too is therefore computationally limited to small examples. On the other hand, a symbolic model checker for dynamic 
epistemic logic called SMCDEL exists, and contains symbolic representations for various logics, including the \textit{synchronous} Gossip Problem (CITE).
This tool is much more general, and can be used for many logic problems and puzzles, however in terms of the Gossip Problem, 
only a symbolic model checker for the synchonous gossip problem exists, and due to keeping track of the 
higher-order logic and distinctions between calling and secrets, this implementation can take a simple model and 
blow it up to check simple formulas. 

This project expands on SMCDEL's functionality. To begin with, in Section X we start by implementing
some functions for interpretting the current state which makes Gossip Problems in SMCDEL, as well as
our future work, easier to interpret. Next, in Section Y we use the definition of knowledge transformers provided in SMCDEL to create
knowledge transformers for the \textit{transparent} Gossip Problem. 

Finally, in Section Z we make a simpler knowledge transformer, which cuts down on the complexity of 
computing the synchronous gossip problem, with the tradeoff of losing higher-order knowledge.  

-- explain  
-- transparent
-- simple


\section{Background}
