\section{Introduction}

\textit{The Gossip Problem} or \textit{Gossip} is the problem of sharing information in a network. Many variants of Gossip exist, each with their own computational challenges. Most notably, a distinction is made between the situation where agents know which agents share information and when, known as the \textit{Transparent} Gossip Problem and when agents only know that information is shared, known as the \textit{Synchronous} Gossip Problem.

For modelling Gossip, an explicit model checker for Gossip called \textit{GoMoChe} exists \cite{gattinger2023gomoche}. Explicit model checkers are generally less efficient than symbolic ones, which aim to cut down on computation time. GoMoChe too is therefore computationally limited to small examples. On the other hand, a symbolic model checker for dynamic epistemic logic called \textit{SMCDEL} exists, and contains symbolic representations for various logics, including Gossip \cite{GattingerThesis2018}. This tool is much more general, and provides a framework for modelling many logic problems and puzzles. However in terms of Gossip, only a symbolic model checker for the Synchronous Gossip Problem exists, and due to keeping track of the higher-order knowledge between agents, this model can blow up in terms of complexity.

This project expands on SMCDEL's functionality. To begin with, in Section \ref{sec:Background} we will start with an overview of how Gossip is modelled in SMCDEL, and specifically explaining the so-called \textit{Knowledge Transformer}, and specifically how it is used to provide updates to the state in \cite{GattingerThesis2018}. Next, in Section \ref{sec:Explain} we start by implementing some functions for interpretting the current state, which makes the Synchronous Gossip Problem already provided in SMCDEL easier to interpret, as well as our future work. From here, we create Knowledge Transformers for the Transparent Gossip Problem in Section \ref{sec:Transparent}. To conclude our work, in Section \ref{sec:Simple} we make a \textit{Simple} Knowledge Transformer, which cuts down on the complexity of computing the Synchronous Gossip Problem, with the tradeoff of losing higher-order knowledge. Finally, the code of Section \ref{sec:Explain}, \ref{sec:Transparent}, and \ref{sec:Simple} is tested in Subsections \ref{ssec:ExplainTests}, X (see fixme), and \ref{ssec:SimpleTests} respectively.

%% fixme: where do we test Transparent
%% fixme: how does ClassicTransformerSpec, and simpletests fit into this ?? 


