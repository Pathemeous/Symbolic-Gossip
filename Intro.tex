\section{Introduction}

\textit{The Gossip Problem} or historically \textit{the Telephone Problem} is the problem of sharing information in a network.
A set of $n$ agents, each with their own secret, may execute one-on-one calls between each other in which they share all secrets that they know.
Agents share all secrets in a call, and nothing else than secrets. In particular, they share no higher order information.
The goal of the problem is to find a (minimal) sequence of calls so that all agents are \emph{experts}, i.e. they know all $n$ secrets.

Various variants of Gossip exist, each with their own computational challenges.
Most notably, a distinction is made between the \textit{Transparent} setting - the situation where all agents know
which agents exchange information at any update - and the \textit{Synchronous} setting, where agents know when an
update occurs but not which agents exchange information during that update.

For modelling Gossip, an explicit model checker for Gossip called \textit{GoMoChe} exists \cite{gattinger2023gomoche}.
Explicit model checkers are generally less efficient than symbolic ones, which aim to cut down on computation time.
GoMoChe too is therefore computationally limited to small examples. On the other hand,
a symbolic model checker for dynamic epistemic logic (DEL) called SMCDEL exists, which is much more general than \textit{GoMoChe}.
SMCDEL implements both \textbf{K} and \textbf{S5} knowledge structures and contains symbolic representations for various logic problems,
including Gossip \cite{GattingerThesis2018}.

However, the Gossip implementation in SMCDEL only covers the synchronous version.
Moreover the \textbf{S5} symbolic structure and transformer updates have an inherent weakness
as they must duplicate parts of and increase the size of the knowledge structure with each update.
The Gossip transformer implementation from \cite{GattingerThesis2018} quickly causes a blowup of the symbolic knowledge structure.
This even leads to sometimes worse runtime performance than the explicit model checker \cite{danielMasterThesis}.

A solution to the transformation method and its inherent knowledge structure growth was proposed in the unpublished master's thesis by \cite{danielMasterThesis},
who introduces the notion of a \textit{Simple Knowledge Transformer} that can replace the \textit{Classic Knowledge Transformer} in certain epistemic settings.
An existing implementation by \cite{HaitianHanabi} extends SMCDEL to incorporate updates with Simple Transformers,
but the translation from a classical transformer to a simple one is not always obvious and none exists for Gossip yet.

This project expands on SMCDEL's functionality. Section \ref{sec:Background} contains a description of
the Classic Knowledge Transformer in SMCDEL, and specifically how it is used to model updates to the state in \cite{GattingerThesis2018}.
Next, Section \ref{sec:Explain} contains a number of functions that provide an interpretation of the current state,
which makes the Synchronous Gossip Problem already provided in SMCDEL more user-friendly.
Next we create a variant of the Classic Transformer for the transparent variant of the Gossip Problem in Section \ref{sec:Transparent}.
To conclude our work, Section \ref{sec:Optimization} describes our implementation of the Simple Transformer,
which cuts down on the complexity of computing the Synchronous Gossip Problem, with the tradeoff of losing higher-order knowledge.
The test results of are discussed in \ref{ssec:ExplainTests} and the various transformer implementations are benchmarked in Section \ref{sec:Benchmarks}.
