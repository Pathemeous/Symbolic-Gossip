\section{Introduction}

\textit{The Gossip Problem} or \textit{Gossip} is the problem of sharing information in a network. 
Many variants of Gossip exist, each with their own computational challenges. 
Most notably, a distinction is made between the \textit{Transparent} Gossip Problem - the situation where all agents know 
which agents exchange information at any update - and the \textit{Synchronous} Gossip Problem, where agents know when an 
update occurs but not which agents exchange information during that update.

For modelling Gossip, an explicit model checker for Gossip called \textit{GoMoChe} exists \cite{gattinger2023gomoche}. 
Explicit model checkers are generally less efficient than symbolic ones, which aim to cut down on computation time. 
GoMoChe too is therefore computationally limited to small examples. On the other hand, 
a symbolic model checker for dynamic epistemic logic (DEL) called SMCDEL exists, which is much more general than \textit{GoMoChe}. 
SMCDEL is implemented for
both $K$ and $S5$ and contains symbolic representations for various logic problems, including Gossip \cite{GattingerThesis2018}. 
However, in terms of Gossip, SMCDEL only covers an encoding of the Synchronous Gossip 
Problem (in standard S5 DEL), and the implementation of its update function causes the model to blow up in terms of 
complexity.

A solution to this exponential blowup was proposed in the unpublished master's thesis by \cite{danielMasterThesis}, in the shape 
of a \textit{Simple Knowledge Transformer} that should replace the \textit{Classic Knowledge Transformer} from SMCDEL. An existing 
implementation by (TODO HAITIAN) extends SMCDEL to incorporate updates with Simple Transformers, but an instance of this transformer
tailored to the Gossip problem wasn't included. 

This project expands on SMCDEL's functionality. Section \ref{sec:Background} contains a description of 
the Classic Knowledge Transformer in SMCDEL, and 
specifically how it is used to model updates to the state in \cite{GattingerThesis2018}. 
Next, Section \ref{sec:Explain} contains a number of functions that provide an interpretation of the current state, 
which makes the Synchronous Gossip Problem already provided in SMCDEL more user-friendly. 
Next we create a variant of the Classic Transformer for the transparent variant of the Gossip Problem in Section \ref{sec:Transparent}. 
To conclude our work, Section \ref{sec:Simple} describes our implementation of the Simple Transformer, 
which cuts down on the complexity of computing the Synchronous Gossip Problem, with the tradeoff of losing higher-order knowledge. 
Finally, the code of Section \ref{sec:Explain}, \ref{sec:Transparent}, and \ref{sec:Simple} is tested in Subsections \ref{ssec:ExplainTests}, 
X (see fixme), and \ref{ssec:SimpleTests} respectively.

%% fixme: where do we test Transparent
%% fixme: should we mention/cite Haitian somewhere??
