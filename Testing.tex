\section{Testing}\label{sec:Testing}
We now discuss the results of the test suite written in \texttt{Hspec}.
As this project relies heavily on the implementation of SMCDEL to execute updates to knowledge scenes,
we assume the correctness of its code and only focus on the correctness of the transformer implementations for Gossip,
as well as the correctness of the \texttt{gsi} function to decode propositions.

In this section we do not show all test cases, and only highlight the most interesting tests. For the full source code, please
refer to the appendix\ref{app:Tests}.

The test coverage is near 100\%. The only omissions are from functions returning \texttt{IO ()} type.
The coverage report can be found in the GitHub repository.

\subsection{GSI Tests}
The GSI suite is tested for correct and decoding of the propositions.
Please note that the IO functions are not tested and only the pure functions are.

\subsection{Gossip Transformer Test Suite}
We developed a set of test cases that verifies the semantics of the gossip problem.
The tests can be divided into roughly three sections:
\begin{itemize}
    \item factual change of secret atoms (call effects)
    \item direct knowledge effects of agents directly invovled in the call
    \item inferred knowledge effects of calls that agents were not directly involved in
\end{itemize}

The factual change is identical regardless of the type of semantics used, which allows the tests for them to be used for all transformers.
The knowledge tests moreover assume (at least) synchronous semantics, which conveniently means they hold for the transparent case too.
We provide additional knowledge tests that hold only for the transparent semantics, and ascertain their negative result in the synchronous setting.

The `inferred` knowledge tries to isolate cases where agents should learn about the (effects of) calls that they were not involved in.
A typical case is with a synchronous setting of three agents $a$, $b$, and $c$:
if the first call $(a,b)$ happens, then $c$ can observe that some call happened due to synchronicity.
However, as $c$ knows she was not invovled herself, and knows there are only two other agents,
she must therefore conclude that the call was between $(a,b)$.
Agent $c$ then knows already that $S_ab \land S_ba$ holds, without ever learning this in a gossip call herself.

\subsubsection{Bugs in SMCDEL Implementation}
Using our test suite we managed to find a bug in the classical transformer as implemented in SMCDEL. In particular the following case fails:
\begin{showCode}
    it "Knowledge: agent knows knowledge of other agent in same call" $ do
    eval (after 4 [(0,1),(1,2)]) (K "2" (has 4 1 0)) `shouldBe` True
    it "Knowledge (inferred): third agent infers knowledge of first agent after 2 calls (4 agents)" $ do
    eval (after 4 [(0,1),(1,2)]) (K "2" (has 4 0 1)) `shouldBe` True
\end{showCode}
We observe that the simple transformer implementation of the same semantics is able to satisfy the first test.
It does fail the second test, but this should be caused by a limitation in its implementation.
The transparent transformer that we defined in \ref{sec:Transparent} satisfies both tests.

\subsubsection{Limitations of the Simple transformer}
The simple transformer implementation fails all `inferred' knowledge tests.
This makes sense as the only knowledge effects that can happen is when an agent is in a call and their observables are updated.
Any other effect taking place, will only be revealed to the agent if they are in a call that shares these secret atoms.

As noted before, this limitation seems to only restrict the knowledge of agents, and not lead to false knowledge.